%%%%%%%%%%%%%%%%%%%%%%%%%%%%%%%%%%%%%%%%%%%%%%%%%%%%%%%%%%%%%
%% Based on a TeXnicCenter-Template, which was             %%
%% created by Christoph Börensen                           %%
%% and slightly modified by Tino Weinkauf.                 %%
%%                                                         %%
%% Then, a third guy - me - put in some modifications.     %%
%%                                                         %%
%% IFT2245 - Rapport TP1                                   %%
%%%%%%%%%%%%%%%%%%%%%%%%%%%%%%%%%%%%%%%%%%%%%%%%%%%%%%%%%%%%%

\documentclass[letterpaper,12pt]{scrartcl}
% Optimised for letter. Add ",twosides" to use the two-sides layout.

% Margins
    \usepackage{vmargin}
    \setpapersize{USletter}
    \setmargins{2.0cm}%	 % Left edge
               {1.5cm}%  % Top edge
               {17.7cm}% % Text width
               {21.0cm}% % Text height
               {14pt}%	 % Header height
               {1cm}%    % Header distance
               {0pt}%	 % Footer height
               {2cm}%    % Footer distance
				
% Graphical bugfix (about footnotes)
    \usepackage[bottom]{footmisc}

% Fonts and locale
	\usepackage{t1enc}
	\usepackage[utf8]{inputenc}
	\usepackage{times}
	\usepackage[francais]{babel}
	\usepackage{amsmath}

	\AtBeginDocument {%
	    \renewcommand\tablename{\textsc{Tableau}}
	}

% Graphics
	\usepackage[pdftex]{graphicx}
	\usepackage{color}
	\usepackage{eso-pic}
	\usepackage{everyshi}
	\renewcommand{\floatpagefraction}{0.7}

% Enable hyperlinks
	\usepackage[pdftex=true]{hyperref}
	
% Table layout
	\usepackage{booktabs}

% Caption
	\usepackage{ccaption}
	\captionnamefont{\bf\footnotesize\sffamily}
	\captiontitlefont{\footnotesize\sffamily}
	\setlength{\abovecaptionskip}{0mm}

% Header and footer settings
	\usepackage{scrpage2} 
	\renewcommand{\headfont}{\footnotesize\sffamily}
	\renewcommand{\pnumfont}{\footnotesize\sffamily}

% Pagestyles
	\defpagestyle{cb}{
		(\textwidth,0pt) % Sets the border line above the header
		{\pagemark\hfill\headmark\hfill} % Doublesided, left page
		{\hfill\headmark\hfill\pagemark} % Doublesided, right page
		{\hfill\headmark\hfill\pagemark} % Onesided
		(\textwidth,1pt)} % Sets the border line below the header
		{(\textwidth,1pt) % Sets the border line above the footer
		{{\it Rapport TP1 (IFT2245)}\hfill Sulliman Aïad et François Poitras} % Doublesided, left page
		{Charles Langlois et François Poitras\hfill{\it Rapport TP1 (IFT2245)}} % Doublesided, right page
		{Charles Langlois et François Poitras\hfill{\it Rapport TP1 (IFT2245)}} % One sided printing
		(\textwidth,0pt) % Sets the border line below the footer
	}

% Empty pages style
	\renewpagestyle{plain}
		{(\textwidth,0pt)
			{\hfill}{\hfill}{\hfill}
		(\textwidth,0pt)}
		{(\textwidth,0pt)
			{\hfill}{\hfill}{\hfill}
		(\textwidth,0pt)}

% Footnotes
	\renewcommand{\footnoterule}{\rule{5cm}{0.2mm} \vspace{0.3cm}}
	\deffootnote[1em]{1em}{1em}{\textsuperscript{\normalfont\thefootnotemark}}

\pagestyle{plain}

\begin{document}
	\begin{center}
		\vspace{2cm}

		{\Huge\bf\sf Rapport du Travail Pratique 1}

		\vspace{0.5cm}

		{\bf\sf (TP1)}

		\vspace{4cm}

		{\bf\sf Par}

		\vspace{0.5cm}{\large\bf\sf Charles Langlois et François Poitras}

		\vspace{2cm}

		{\bf\sf Rapport présenté à}

		\vspace{0.5cm}{\large\bf\sf M. Stefan Monnier}

		\vspace{2cm}

		{\bf\sf Dans le cadre du cours de}

		\vspace{0.5cm}{\large\bf\sf Systèmes d'exploitation (IFT2245)}

		\vspace{\fill}
		\today

		\vspace{0.5cm}Université de Montréal
	\end{center}
	
	\newpage

	\pagestyle{cb}
	
	\tableofcontents

	\newpage

	\section{Fonctionnement du shell}
			Le programme démarre en affichant le répertoire de travail courant, qui correspond au répertoire d'où a été exécuté le programme \textit{ch}. À partir de ce point, toutes les commandes disponibles dans les dossiers présent dans la variable d'environnement \textit{PATH} peuvent êtres éxecutées. Il est aussi possible de rediriger la sortie ou l'entrée d'une commande en utilisant le symbole '>' ou '<', comme dans la plupart des terminaux Unix. De la même façon, l'usager peut transmettre la sortie d'une commande à une autre en utlisant des \textit{pipes} dénotés par le symbole '|' entre les commandes, ainsi qu'enchainer plusieurs commandes sur une même ligne en les séparant par des point-virgules(';'); 

			Il est possible d'utiliser une astérisque pour faire référence à l'ensemble des fichiers et dossiers du répertoire courant dont le nom ne commence pas par '.'(excluant donc les dossiers symboliques ``.'' et ``..''. Par exemple, la commande \textit{echo *} affichera le nom de tous les fichiers et dossiers du répertoire courant. L'utilisateur peut en tout tempsquitter le programme en utilisant la commande \textit{quit} ou encore en envoyant le signal de fin de fichier(end-of-file) par CTRL-D.

		    \subsection{Traitement de l'entrée}
            La lecture de l'entré se fait commande par commande, dans une boucle itérative de style \textit{read-execute-print}.
            
            Une commande est délimité par les caractères ';' ,'|' et '\textbackslash n'(fin de ligne).
            
            La lecture d'une commande se fait par la fonction \textit{readCommand}, qui lit une série de 'jetons'(une séquence de caractères séparé par des espaces) jusqu'à rencontrer un des caractères délimiteurs.
            Une fois la commande lue et stockée dans un buffer, elle est divisée en ses \textit{jetons} constituants. Ce processus élimine les espaces au début et à la fin de la commande.
            Si la commande lue est ``quit'', le reste du traitement est sauté, le reste de la ligne est ignorée, la boucle est intérompue et l'exécution du \textit{shell} se termine
            Si la liste de jetons est vide, le reste du traitement est sauté, le reste de la ligne est ignoré, l'invite de commande(``path\%\%'' ) et une nouvelle commande est attendue.                        
            Sinon, si le caractère délimitant la commande est ';' ou '\textbackslash n', la commande est exécutée, avec les entré et sortie précédemment désignées(par défaut, ceux-ci correspondent aux \textit{file descriptors} '0' et '1', correspondant au \textit{stdin} et \textit{stdout}).
            Si le caractère délimitant est plutôt un caractère de redirection('<','>'), le prochain jeton lu devrait être un nom de fichier, et l'entré('<') ou la sortie('>') est changée pour un descripteur de fichier correspondant à ce fichier.
            Si le caractère délimitant est le caractère de ``plomberie'' '|', le puit de sortie est changé pour le descripteur de fichier généré par la fonction \textit{pipe}. 
            La commande est par la suite exécutée.
            
            Si la commande exécutée se terminait sur une fin de ligne, l'entré et la sortie sont remises au \textit{stdin} et \textit{stdout}, et l'invite de commande est réimprimée, et 
            Si la commande exécutée se terminait par '|', l'entré est changée pour un descripteur de fichier généré par le dernier appel à \textit{pipe}.
            Pour tout autre fin de commande(';'), l'entré et la sortie sont remises au \textit{stdin} et \textit{stdout}, et une nouvelle commande est lue à la prochaine itération.     
			
			\subsubsection{Commandes externes}
			L'argument \textit{args} passé à la fonction \textit{execCommand} est un tableau de chaînes de charactères, soit de type \textit{char**}, avec \textit{args[0]} correspondant à la commande entrée. On vérifie s'il s'agit de 'cd'. Si oui, la fonction du même nom est appelée et gère le changement de dossier.

            Pour tout autre commande, on assume qu'il s'agit d'une commande externe. Le processus doit alors être dupliqué à l'aide de \textit{fork()}.

			Par la suite, une vérification du \textit{PID} doit être faite pour permettre de savoir si le code suivant sera exécuté dans le processus enfant ou parent. Si on est dans le processus enfant, on vérifie les descripteurs de fichiers passés en argument et on appelle \textit{execvp} pour exécuter le programme. L'utilisation particulière de cette commande permet d'éviter de gèrer la variable d'environnement \textit{PATH}, en raison du suffixe 'p'. De plus, la famille de fonctions \textit{exec} qui comporte un 'v' en suffixe demande de recevoir un \textit{char**}(tableau de \textit{strings}) en guise d'arguments aux programmes externes. Comme c'est ce que nous passons en argument à notre fonction, nous n'avons aucun traitement particulier à faire. Dans le cas où l'exécution est dans le processus parent, il s'agit simplement d'attendre que le processus enfant termine et de retourner le status de l'enfant, s'il y a lieu.

			\subsubsection{Commandes internes}
			Par manque de temps, les seules commandes internes qui sont gérées par notre shell sont 'cd' et 'quit'. La deuxième ayant été expliquée plus haut, nous allons ici décrire le comportement de la première. 'cd' se comporte comme nous sommes habitués avec les autres terminaux unix. Comme il en est traditionnellement, le caractère '~' est interprété comme le dossier 'home'(i.e. ``/home/<\textit{utilisateur}>/''), et '-' indique le dernier répertoire accèdé(initialisé au répertoire duquel le \textit{shell} est exécuté).

		\subsection{Gestion des erreurs}
			Les messages d'erreurs sont des citations de Margaret Tatcher modifiées selon le context. Les messages d'erreurs restent descriptifs des problèmes rencontrés. Par exemple, si on entre une commande qui n'existe pas, le shell nous répondra «There's no such thing as <command>. Only families and individuals.» où <command> sera le nom de la commande entrée. D'autres messages sont générés si l'utilisateur tente d'accèder à un endroit sur le disque où il n'a pas accès, ou encore si la commande entrée est trop grande. 

	        \section{Problèmes rencontrés}
		Puisque nous parlons de programmation en C, l'un des principaux problèmes est bien évidemment la gestion manuelle de la mémoire. Les \textit{segmentation faults} furent nombreux au cours du développement et parfois difficiles à règler. Par exemple, l'expansion d'arguments a momentanément posé problème puisqu'il fallait pouvoir prédire correctement le nombre de chaines de caractères(\textit{char*}) à allouer, et donc le nombre de jetons composant la commande après l'expansion des ``métacaractères'' comme l'astérique, et aussi remplacer en mémoire l'astérisque par les noms des fichiers et dossiers présents dans le répertoire courant(excluant ceux commençant par '.'). 

		L'autre problème concerne notre connaissance des fonctions POSIX. En effet, comme nous n'étions pas en connaissance de ces fonctions et de leur fonctionnement, nous avons dû nous documenter sur leur fonctionnement. Dans les premières versions de notre code, nous ne connaissions pas toutes les fonctions de la famille \textit{exec} et avons donc commencé à développer des fonctions pour \textit{parser} le contenu de la variable d'environnement \textit{PATH}. Nous avons compris comment utiliser les fonctions de gestions de \textit{file descriptors} d'une façon similaire, en consultant les pages de manuel des fonctions appropriées, ainsi qu'en consultant \textit{StackOverflow}. Ces fonctions se sont finalement avérées très simples à utiliser.

        \section{Sources}
        fonction \textit{countDirectoryContent}: 
       À cette fonction, nous avons rajouté la gestion d'erreurs, qui était absente dans la réponse acceptée.


        \url{http://stackoverflow.com/questions/1121383/counting-the-number-of-files-in-a-directory-using-c}
        
        fonction \textit{tokenize}: 
        À cette fonction, nous avons rajouté la gestion de l'expansion d'arguments.


        \url{http://stackoverflow.com/questions/8106765/using-strtok-in-c}
        
\end{document}
