%%%%%%%%%%%%%%%%%%%%%%%%%%%%%%%%%%%%%%%%%%%%%%%%%%%%%%%%%%%%%
%% Based on a TeXnicCenter-Template, which was             %%
%% created by Christoph Börensen                           %%
%% and slightly modified by Tino Weinkauf.                 %%
%%                                                         %%
%% Then, a third guy - me - put in some modifications.     %%
%%                                                         %%
%% IFT2245 - Rapport TP1                                   %%
%%%%%%%%%%%%%%%%%%%%%%%%%%%%%%%%%%%%%%%%%%%%%%%%%%%%%%%%%%%%%

\documentclass[letterpaper,12pt]{scrartcl}
% Optimised for letter. Add ",twosides" to use the two-sides layout.

% Margins
    \usepackage{vmargin}
    \setpapersize{USletter}
    \setmargins{2.0cm}%	 % Left edge
               {1.5cm}%  % Top edge
               {17.7cm}% % Text width
               {21.0cm}% % Text height
               {14pt}%	 % Header height
               {1cm}%    % Header distance
               {0pt}%	 % Footer height
               {2cm}%    % Footer distance
				
% Graphical bugfix (about footnotes)
    \usepackage[bottom]{footmisc}

% Fonts and locale
	\usepackage{t1enc}
	\usepackage[utf8]{inputenc}
	\usepackage{times}
	\usepackage[francais]{babel}
	\usepackage{amsmath}

	\AtBeginDocument {%
	    \renewcommand\tablename{\textsc{Tableau}}
	}

% Graphics
	\usepackage[pdftex]{graphicx}
	\usepackage{color}
	\usepackage{eso-pic}
	\usepackage{everyshi}
	\renewcommand{\floatpagefraction}{0.7}

% Enable hyperlinks
	\usepackage[pdfborder=000,pdftex=true]{hyperref}
	
% Table layout
	\usepackage{booktabs}

% Caption
	\usepackage{ccaption}
	\captionnamefont{\bf\footnotesize\sffamily}
	\captiontitlefont{\footnotesize\sffamily}
	\setlength{\abovecaptionskip}{0mm}

% Header and footer settings
	\usepackage{scrpage2} 
	\renewcommand{\headfont}{\footnotesize\sffamily}
	\renewcommand{\pnumfont}{\footnotesize\sffamily}

% Pagestyles
	\defpagestyle{cb}{
		(\textwidth,0pt) % Sets the border line above the header
		{\pagemark\hfill\headmark\hfill} % Doublesided, left page
		{\hfill\headmark\hfill\pagemark} % Doublesided, right page
		{\hfill\headmark\hfill\pagemark} % Onesided
		(\textwidth,1pt)} % Sets the border line below the header
		{(\textwidth,1pt) % Sets the border line above the footer
		{{\it Rapport TP1 (IFT2245)}\hfill Sulliman Aïad et François Poitras} % Doublesided, left page
		{Charles Langlois et François Poitras\hfill{\it Rapport TP1 (IFT2245)}} % Doublesided, right page
		{Charles Langlois et François Poitras\hfill{\it Rapport TP1 (IFT2245)}} % One sided printing
		(\textwidth,0pt) % Sets the border line below the footer
	}

% Empty pages style
	\renewpagestyle{plain}
		{(\textwidth,0pt)
			{\hfill}{\hfill}{\hfill}
		(\textwidth,0pt)}
		{(\textwidth,0pt)
			{\hfill}{\hfill}{\hfill}
		(\textwidth,0pt)}

% Footnotes
	\renewcommand{\footnoterule}{\rule{5cm}{0.2mm} \vspace{0.3cm}}
	\deffootnote[1em]{1em}{1em}{\textsuperscript{\normalfont\thefootnotemark}}

\pagestyle{plain}

\begin{document}
	\begin{center}
		\vspace{2cm}

		{\Huge\bf\sf Rapport du Travail Pratique 1}

		\vspace{0.5cm}

		{\bf\sf (TP1)}

		\vspace{4cm}

		{\bf\sf Par}

		\vspace{0.5cm}{\large\bf\sf Charles Langlois et François Poitras}

		\vspace{2cm}

		{\bf\sf Rapport présenté à}

		\vspace{0.5cm}{\large\bf\sf M. Stefan Monnier}

		\vspace{2cm}

		{\bf\sf Dans le cadre du cours de}

		\vspace{0.5cm}{\large\bf\sf Systèmes d'exploitation (IFT2245)}

		\vspace{\fill}
		\today

		\vspace{0.5cm}Université de Montréal
	\end{center}
	
	\newpage

	\pagestyle{cb}
	
	\tableofcontents

	\newpage

	\section{Fonctionnement du shell}
			Le programme démarre en affichant le \textit{path} vers le répertoire de travail courant, qui correspond au répertoire où le programme \textit{ch} est situé. À partir de ce point, toutes les commandes disponibles qui sont définies dans la variable d'environnement \textit{PATH} peuvent êtres éxecutées, avec leurs arguments respectifs. Il est aussi possible de rediriger la sortie ou l'entrée d'une commande en utilisant le symbole '>' ou '<', comme dans la plupart des terminaux sous Linux. De la même façon, l'usager peut utliser des \textit{pipes} avec le symbole '|'. Il est possible d'utiliser une astérisque pour faire référence à tous les fichiers normaux --- c'est-à-dire tous les fichiers sauf '.' et '..' --- dans le répertoire courant. Par exemple, la commande \textit{echo *} affichera le nom de tous les fichiers du répertoire courant. Entre deux entrées, l'utilisateur peut quitter le programme en utilisant la commande \textit{quit} ou encore en utilisant CTRL-D.
		\subsection{Traitement de l'entrée}
			Notre shell commence par éliminer tous les espace au début de la commande, s'il y en a. Ensuite, on lit charactère par charactère jusqu'à ce qu'un caractère spécial soit atteint. Ces charactères sont le retour à la ligne, les opérateurs de redirection, de pipe, un espace ou la fin du fichier (\textit{EOF}).
			
			\subsubsection{Commandes externes}
			L'utilisation de execvp blablabla...
			\subsubsection{Commandes internes}
				Par manque de temps, les seules commandes internes qui sont gérées par notre shell sont 'cd' et 'quit'. La deuxième ayant été expliquée plus haut, nous allons ici discuter du comportement de la première. 'cd' se comporte comme nous sommes habitués, à l'exception de la gestion du '~'. En effet, il est possible de faire 'cd' et toutes ses variations définies dans la plupart des terminaux Linux, mais les variations comme \textit{cd ~/machin} ne sont pas gérées. Il est possible d'utiliser \textit{cd -} pour revenir au dossier précédent.
		\subsection{Gestion des erreurs}
			Les messages d'erreurs sont des citations inspirées de Margaret Tatcher. Même si cela peut paraître ridicule, les messages
		d'erreurs sont quand même descriptifs du problème qui c'est produit. On arrive sans problème à comprendre ce qui c'est passé. Par exemple, si on entre une commande qui n'existe pas, le shell nous répondra «There's no such thing as <command>. Only families and individuals.» où <command> sera le nom de la commande entrée.
	\section{Problèmes rencontrés}
		Puisque nous parlons de programmation en C, l'un des principaux problèmes est bien évidemment la gestion de la mémoire. Les \textit{segmentation fault} furent nombreux au cours du développement et parfois subtils.
\end{document}